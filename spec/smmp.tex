\documentclass[%
%reprint,
%superscriptaddress,
%groupedaddress,
%unsortedaddress,
%runinaddress,
%frontmatterverbose, 
preprint,
%showpacs,preprintnumbers,
%nofootinbib,
%nobibnotes,
%bibnotes,
amsmath,amssymb,
aps,
%pra,
prb,
%rmp,
%prstab,
%prstper,
floatfix,
]{revtex4-1}


\usepackage{graphicx}% Include figure files
%\usepackage{dcolumn}% Align table columns on decimal point
\usepackage{bm}% bold math
\usepackage{amsmath}
\usepackage{subfigure}

\begin{document}

\preprint{Version 0.3}

\title{SMMP: A Secure Multi-Party Messaging Protocol}% Force line breaks with \\

\author{David R. Andersen}
 \email{k0rx@uiowa.edu}
 \altaffiliation[Also at ]{Department of Physics and Astronomy, The University of Iowa.}%Lines break automatically or can be forced with \\
\author{Mark S. Andersland}
 \affiliation{Department of Electrical and Computer Engineering, The University
 of Iowa, Iowa City, IA 52242}
\author{Tycho J. Andersen}
 \affiliation{Canonical, Inc.\\ Madison, WI 53703}

\date{\today}% It is always \today, today,
             %  but any date may be explicitly specified

\begin{abstract}
We describe a new secure, multi-party, synchronous communication protocol.
The protocol follows a peer-to-peer model and provides perfect forward
secrecy, perfect future secrecy, and plausible deniability for participants in
the conversation, as long as at least two participants are honest.
The protocol uses the elliptic-curve Diffie-Hellman key agreement
protocol to generate a set of shared secrets between a group of
participants. The group is initially formed by an Organizer.
The Organizer role is abandoned when key agreement is accomplished, leaving
the original organizer as a one of $N$ peers in the messaging network.
All setup operations prior to key agreement can take place over an insecure
channel.
\end{abstract}

\pacs{73.20.Mf}% PACS, the Physics and Astronomy
                             % Classification Scheme.
%\keywords{Suggested keywords}%Use showkeys class option if keyword
                              %display desired
\maketitle

%\tableofcontents
\section{\label{sec:Introduction}Introduction}
Secure multi-party messaging has been an elusive goal of security researchers
for a long time. The initial `virtual server' approach\cite{ref:bian}
was deemed unsatisfactory
because it has the unfortunate drawback of not permitting all participants to
confirm that they were receiving unmodified copies of all messages.
A further attempt to solve this problem\cite{ref:goldberg} was incomplete
because of the lack of a simple key-agreement strategy. Work on the problem
continues to date, most notably with\cite{ref:cryptocat}, however the
key-agreement problem remains unsolved.

In this paper, we describe a novel secure multi-party messaging protocol with a
simple key-agreement algorithm that provides perfect forward secrecy (PFS),
perfect future secrecy (PFuS),
and plausible deniability (PD) for participants in a group conversation. Our protocol
is a peer-to-peer protocol, whereby each
participant periodically contributes
key material to the group key set. The group is
initially constituted by an Organizer. Following the formation of the group, the
Organizer role is abandoned and the initial Organizer assumes a peer role, the
same as all other Participants.

\section{\label{sec:keypoints}Key Points}
\begin{itemize}
\item Each group key agreement is facilitated by an Organizer, who is also
Participant $P_0$. Following the key agreement stage, the Organizer is no longer
required and the Participant $P_0$ role is that of a peer like any other
participant.
\item Elliptic curve Diffie-Hellman key operations are performed on curve25519
with 32 byte
public and private keys.
\item The protocol is synchronous. A method for resynchronizing those
participants that lose synchronization due to lost packets from collisions or
transport failure is included in the protocol.
\item Group setup is via an insecure channel.
\item Putative participants authenticate to the Organizer, prior to
activation of the group protocol.
\item Participants may not be added to the group. This would
require a reinitialization of the group secrets. A Participant may leave the
group.
\item The underlying symmetric encryption algorithm used for communication is
unspecified.
\end{itemize}

\section{\label{sec:notation}Notation}
The following notation is used:
\begin{itemize}
\item $N$ is the total number of participants in the group (including the
Organizer)
\item $\oplus$ is the XOR operator
\item $\{N_i\}$ is the set of all $N_i$ and $\{N_i\}_m = N_m$.
\item $\oplus_{j \ne i} \, x_j$ means XOR of all $x_j$ except $x_i$.
\item  $\oplus_i \, x_i$ means XOR of all $x_i$.
\item $\mathcal{\hat{A}}_i$ is the ratchet state for
participant $P_i$.
\item $||$ is the concatenation operator
\item Lower case keys are private ECDH keys.
\item Upper case keys are public ECDH keys.
\item hash() is a secure, one-way cryptographic hash function.
\item hmac$(k,m)$ is a secure hashing message authentication function that
hashes a message $m$ using key $k$.
\item KDF() is a secure key derivation function, \textit{e.g.} pbkdf2.
\item $c = e(k, m)$ and $m = d(k,c)$ are the underlying (unspecified) symmetric
encryption and decryption algorithms that use a key $k$ to transform a plaintext
$m$ into ciphertext $c$ and \textit{vice versa}.
\item MK is a master key from which the initial ratchet state
$\mathcal{\hat{A}}_i$ is computed by each participant $P_i$. This parameter is
securely erased upon computing the initial ratchet state.
\item $X$ is the generator for curve25519 and ECDH() is the Diffie-Hellman
operator on curve25519, \textit{i.e.} for private key $u$, the
corresponding public key is $U = \mathrm{ECDH}(u, X)$. The shared secret between
$y$ and $z$ is then $ \mathrm{ECDH}(y,Z) = \mathrm{ECDH}(z,Y) =
\mathrm{ECDH}(y,\mathrm{ECDH}(z, X))$.
\end{itemize}

\section{\label{sec:protocolstate}Protocol State}
Each Participant $P_i$ will maintain the following variables in persistent
storage:
\begin{itemize}
\item RK - the root key
\item HK - the header key
\item NHK - the next header key
\item mk - the message key
\item $v$ - the group private ratchet key
\item $\{R_i\}$ - the set of public ratchet keys from each participant $P_i$
\item $i$ - Participant $P_i$'s participant index number
\item $N$ - the group size
\item group\_name - the group name (this may be different for each Participant)
\item resync\_required - a flag used to determine if a resynchronization of the
ratchet state is required
\end{itemize}

\section{\label{sec:keyagreement}Key Agreement}
The Organizer and Participants complete the following steps:
\begin{enumerate}
\item Participant $P_0$ establishes himself as the group Organizer $O$ and solicits
keys from other potential participants.
\item Organizer $O$ determines the total number of participants $N$.
\item Organizer $O$ assigns a participant index number $i$ to each
participant.
\item Organizer $O$ generates elliptic curve Diffie-Hellman (ECDH) persistent group
identity, and ephemeral group handshake
keys $(u, U )$ and $(w, W)$.
\item Organizer $O$ forwards $U$, $W$, $N$, and $i$ to each participant $P_i$.
\item Each participant $P_i$ generates ECDH
persistent identity, ephemeral handshake and ephemeral ratchet keys $(b_i, B_i)$,
$(k_i, K_i)$ and $(r_i, R_i)$.
\item Each participant $P_i$ forwards $B_i$ and $K_i$ to Organizer $O$.
\item Each participant $P_i$ forwards $R_i$ to all other participants in the
group.
\item Organizer $O$ authenticates the identity of each participant $P_i$ with
identity key $B_i$.
\item Each participant $P_i$ authenticates the group identity $U$ with Organizer
$O$.
\item Organizer $O$ computes:\\
$L_i$ = $\mathrm{hash}(\mathrm{ECDH}(u, K_i) \,
|| \, \mathrm{ECDH}(w, B_i) \, || \, \mathrm{ECDH}(w, K_i))$,\\
$G_i = \mathrm{hash}(\mathrm{ECDH} (w, R_i)) \, \oplus \, \oplus_{j \ne i} \, L_j$.
\item Organizer $O$ forwards $G_i$ to participant $P_i$ for all participants.
\item Participant $P_i$ computes:\\
$\mathrm{MK} = \mathrm{hash}(\mathrm{ECDH}(k_i, U) \, ||
\, \mathrm{ECDH}(b_i, W) \, || \, \mathrm{ECDH}(k_i, W) \oplus \, G_i \, \oplus
\, \mathrm{hash}(\mathrm{ECDH}(r_i, W))$
\item Participant $P_i$ computes his/her initial ratchet state $\mathcal{\hat{A}}_i$:\\
$\mathrm{RK} = \mathrm{KDF}(\mathrm{MK}, 0\mathrm{x}00)$, \\
$\mathrm{HK} = \mathrm{KDF}(\mathrm{MK}, 0\mathrm{x}01)$, \\
$\mathrm{NHK} = \mathrm{KDF}(\mathrm{MK}, 0\mathrm{x}02)$, \\
$\mathrm{mk} = \mathrm{KDF}(\mathrm{MK}, 0\mathrm{x}03)$, \\
$v = \mathrm{KDF}(\mathrm{MK}, 0\mathrm{x}04)$, \\
$\{R_i\}$ from participants, \\
$N$ from Organizer $O$, \\
$i$ from Organizer $O$, \\
resync\_required = \textit{False}.
\item The Organizer role is complete and all Organizer data
is securely erased by participant $P_0$.
\end{enumerate}

\section{\label{sec:sending}Sending Messages}
Participants wanting to send a message will proceed as follows:
\begin{enumerate}
\item When a participant $P_j$ wishes to communicate with the other
participants, he forms a message $m$.
\item Participant $P_j$ generates a new ephemeral ratchet key $R_j^{new}$ and
sets $R_j = R_j^{new}$.
\item Participant $P_j$ computes preliminary ciphertexts $c_h = e(\mathrm{HK}, j
\, || \, R_j^{new})$ and $c_m = e(\mathrm{mk}, m)$. He the forms $c^\prime = c_h
\, || \, c_m$.
\item Participant $P_j$ computes hmac $c_{hmac} = \mathrm{hmac}(v, c^\prime)$.
\item Participant $P_j$ forms $c = c^\prime \, || \, c_{hmac}$.
\item Participant $P_j$ updates his/her ratchet state $\mathcal{\hat{A}}_j$ as
follows:\\
$\mathrm{RK} = \mathrm{hash}(\mathrm{RK} \, || \, \mathrm{ECDH}(v, \oplus_i
\, R_i))$, \\
$\mathrm{HK} = \mathrm{NHK}$, \\
$\mathrm{NHK} = \mathrm{KDF}(\mathrm{RK}, 0\mathrm{x}02)$, \\
$\mathrm{mk} = \mathrm{KDF}(\mathrm{RK}, 0\mathrm{x}03)$.
\item Participant $P_j$ broadcasts $c$ to all participants $P_i$.
\end{enumerate}

\section{\label{sec:receiving}Receiving Messages}
Participants receiving a message will proceed as follows:
\begin{enumerate}
\item Upon receipt of a ciphertext $c$, participants $P_j$ separates $c$ into
$c^\prime$ and $c_{hmac}$ parts.
\item Participant $P_j$ tests if $\mathrm{hash}(v, c^\prime) = c_{hmac}$. If
they do not match, he/she raises a Bad\_HMAC exception and goes no further.
\item Participant $P_j$ splits $c^\prime$ into header $c_h$ and message $c_m$ components
and obtains $q \, || \, R_q^{new} = d(\mathrm{HK}, c_h)$ and $m = d(\mathrm{mk},
c_m)$. If either of these operations fail, he/she raises a Message\_Undecryptable
exception, sets the resync\_required flag \textit{True},
and passes $c^\prime$ to the message-received housekeeping routine described in
Section \ref{sec:receivehousekeeping}.
\item Participant $P_j$ sets $R_q = R_q^{new}$.
\item Participant $P_j$ updates his/her ratchet state $\mathcal{\hat{A}}_i$ as
follows: \\
$\mathrm{RK} = \mathrm{hash}(\mathrm{RK} \, || \, \mathrm{ECDH}(v, \oplus_i
\, R_i))$, \\
$\mathrm{HK} = \mathrm{NHK}$, \\
$\mathrm{NHK} = \mathrm{KDF}(\mathrm{RK}, 0\mathrm{x}02)$, \\
$\mathrm{mk} = \mathrm{KDF}(\mathrm{RK}, 0\mathrm{x}03)$.

\end{enumerate}
\section{\label{sec:housekeeping}Protocol Housekeeping}
Here we list several protocol housekeeping operations that may be necessary.
The list may be extended if other useful operations are identified.

\subsection{\label{sec:receivehousekeeping}Message Received Housekeeping}
Housekeeping messages will be routed based on a one-byte header prepended to the
payload of the housekeeping message. The byte values and their corresponding
housekeeping tasks are (currently there is only 1): \\

\begin{centering}
\begin{tabular}{|c|c|c|}
\hline
Byte Value & Housekeeping Task & Location \\
\hline
$0\mathrm{x}00$ & Resynchronizing the Ratchet State  & Section \ref{sec:receiveresync}\\
\hline
\end{tabular} \\
\end{centering}
\bigskip
To determine the proper routing, the
participant proceeds as follows:
\begin{enumerate}
\item Participant $P_i$ computes $b \, || \, m = d(v, c^\prime)$. He/she then
finds the value corresponding to byte $b$ in the table above, and sends message
$m$ and control to that section.
\end{enumerate}
\subsubsection{\label{sec:receiveresync}Resynchronizing the Ratchet State: $b =
0\mathrm{x}00$}
It is possible, during communication within the group, that a participant's
ratchet state may become unsynchronized. Transport layer failures due to lost
packets or packet collisions can cause a participant to be unsynchronized. If
this is the case, a Message\_Undecryptable exception will be raised
and control of the decryption will be passed here. The receiving participant
will then proceed as follows:
\begin{enumerate}
\item Participant $P_j$ sets $v = m$.
\item Participant $P_j$ updates his/her ratchet state
$\mathcal{\hat{A}}_j$ as follows:\\
$\{R_i\} = \{ \mathrm{hash}(v \, || \, i)\}$, \\
$\mathrm{RK} = \mathrm{hash}(\oplus_i \, R_i \, || \, \mathrm{ECDH}(v, \oplus_i
\, R_i))$, \\
$\mathrm{HK} = \mathrm{KDF}(\mathrm{RK}, 0\mathrm{x}01)$, \\
$\mathrm{NHK} = \mathrm{KDF}(\mathrm{RK}, 0\mathrm{x}02)$, \\
$\mathrm{mk} = \mathrm{KDF}(\mathrm{RK}, 0\mathrm{x}03)$.
\item Participant $P_j$ sets his resync\_required flag to \textit{False}.
\end{enumerate}

\subsection{\label{sec:sendhousekeeping}Message Send Housekeeping}
It may be necessary at some point to send housekeeping messages. This section
describes procedures to be followed in this case.

\subsubsection{\label{sec:sendresync}Resynchronizing the Ratchet State:
resync\_required = True}
When the resync\_required flag is set to \textit{True}, the ratchet state is
unsynchronized, and decrypting further conversation messages is impossible.
A participant wishing to correct this situation should follow the following
procedures:
\begin{enumerate}
\item Participant $P_j$ generates a new group private ratchet key $v^{new}$.
\item participant $P_j$ computes preliminary ciphertext $c^\prime = e(v, 0\mathrm{x}00
\, || \, v^{new})$.
\item Participant $P_j$ computes hmac $c_{hmac} = \mathrm{hmac}(v, c^\prime)$.
\item Participant $P_j$ forms $c = c^\prime \, || \, c_{hmac}$.
\item Participant $P_j$ delays according to a random backoff algorithm
that should be transport-specific and is unspecified here.
\item Participant $P_j$ tests resync\_required to see if it is still
\textit{True}. If not, a resynchronization message was received and no further
action is necessary. If \textit{True}, participant $P_j$ continues with the next
step.
\item Participant $P_j$ broadcasts $c$ to all participants $P_i$.
\item Participant $P_j$ updates his/her ratchet state
$\mathcal{\hat{A}}_j$ as follows:\\
$v = v^{new}$, \\
$\{R_i\} = \{ \mathrm{hash}(v \, || \, i)\}$, \\
$\mathrm{RK} = \mathrm{hash}(\oplus_i \, R_i \, || \, \mathrm{ECDH}(v, \oplus_i
\, R_i))$, \\
$\mathrm{HK} = \mathrm{KDF}(\mathrm{RK}, 0\mathrm{x}01)$, \\
$\mathrm{NHK} = \mathrm{KDF}(\mathrm{RK}, 0\mathrm{x}02)$, \\
$\mathrm{mk} = \mathrm{KDF}(\mathrm{RK}, 0\mathrm{x}03)$.
\item Participant $P_j$ sets his resync\_required flag to \textit{False}.

\end{enumerate}

\begin{thebibliography}{99}
\bibitem{ref:bian} J. Bian, R. Seker, and U. Topaloglu ``Off-the-Record Instant
Messaging for Group Conversation," \textit{IRI '07: Proceedings of Information
Reuse and Integration,} pp. 79-84, IEEE Computer Society, 2007.
\bibitem{ref:goldberg} I. Goldberg, M. D. Van Gundy, B. Ustanoglu, and H. Chen,
``Multi-Party Off-the-Record Messaging," \textit{CSS '09: Proceedings of the
16th ACM Conference on Computer and Communication Security,} pp. 358-368, ACM,
2009.
\bibitem{ref:cryptocat} Cryptocat Messaging Blog
\url{https://github.com/cryptocat/mpotr}, accessed Feb. 14, 2014.
\bibitem{ref:axolotl} Trevor Perrin and Moxie Marlinspike,
\url{https://github.com/trevp/axolotl/wiki/newversion}, accessed Feb. 14, 2014.
\end{thebibliography}
\end{document}
%
% ****** End of file apssamp.tex ******
